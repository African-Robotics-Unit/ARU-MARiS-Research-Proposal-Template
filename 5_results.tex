%*********************************************************************
%	Preliminary results
%*********************************************************************
\chapter{Preliminary results}
\label{ch:results}
%---------------------------------------------------------------------
% Taken from "Scientific Writing by D. Branch Moody"
%---------------------------------------------------------------------
% When first starting to write a paper, I advise students and 
% post-docs to write from the inside out.  This means that the 
% figures, results, methods figure legends are written prior to 
% the introduction or conclusion sections.  This approach has been
% very successful and is based on the logic that the "meat" of 
% the paper is the results, and the authors have to say and 
% understand exactly what the results as the anchor of the paper 
% before the introduction and discussion can be formulated.  
% Also the results section is straightforward explanation of the 
% data so is easy to write and can usually be accomplished in one day. 
%
% Prior to writing the results section, it is best to produce the 
% main figures of the paper.  The results section strictly follows 
% the order of the figures.  Description of the data Figure 1a is 
% always before 1b and Figure 2 comes before Figure 3, etc?  The 
% results are presented in a logical order that introduces the 
% reagents and findings one by one like characters appearing in a 
% play, rather than the chronological order in which the experiments 
% were done.  
%
% Each paragraph in the results follows a fairly set pattern.  
% 1) premise or question 2) reagents used 3) outcome 4) a narrow 
% and conservative interpretation or summary of the facts. 
%
% For beginning writers, the most common weakness is to head 
% straight for the experiment and the result, while sometimes 
% forgetting to state the premise and formally provide the 
% conclusion.  That is, considering the four-step method described 
% above, the common mistake is to leave out step 1, step 4 or 
% both. The reader wants to know why you did the experiment and 
% how you summarize the findings.  Looking at the very same data,
% individual scientists often draw differing opinions so the 
% "Therefore, we conclude that?" statement is important. 
%
% In the results section only very narrow interpretations are 
% allowed. (The exceptions to this rule are the Journal of 
% Immunology Cutting Edge, JEM Brief Definitive Reports, Science, 
% Nature and a few other journals in which the results and 
% discussion are combined. It is important to decide at the outset 
% if you are writing a separate or combined results and discussion 
% format.)  In the discussion section, the inferences and 
% speculations are used to broaden the interpretation.  but it 
% is important to understand that this is a two step process in 
% which the facts and opinions are strictly separated in the 
% results and discussion sections. 
%
% Perhaps the most important guideline for writing the results 
% section is that the tone and language of conclusions match the 
% strength of the data. That is, is each claim might be "proven" 
% through data that isolate a single cause that is "necessary 
% and sufficient" .  Alternatively the data might support weaker 
% but still useful conclusions.  In this case the are data 
% "suggestive of" or "consistent with" the stated conclusion.  
% The most common mistake that weak or inconclusive data to be 
% presented as strong or conclusive.  This outcome is readily 
% recognized by reviewers and gives the impression that the 
% authors are naïve (overly enthusiastic), self serving or dishonest.  
%
% It is okay to report data that are consistent with or suggestive 
% of a conclusion, while acknowledging that the given experiment 
% is not conclusive. In fact, this is often the best that any 
% single experiment can do.  Many times, key conclusions derive 
% from several experiments that work together to suggest a 
% conclusion.  All papers have flaws.  Experienced reviewers are 
% much more likely to pass your paper forward to publication, 
% if you acknowledge the flaws and explain the limits of the 
% data, rather than pretending that the flaws do not exist. 
%
% Often the weakness of the flaw in the first experiment provides
% the rationale to do the second experiment and you can say 
% this in bridge between paragraphs.  For example the phrase  
% "To address this potential artifact?" is a useful bridge to 
% the next paragraph, as the weakness of one experiment 
% becomes the premise of the next experiment.   Other Connecting 
% words at the beginning of each paragraph are "Next?"  
% "To investigate this further?"  "To determine the mechanism?" 
%---------------------------------------------------------------------


%*********************************************************************
% End of Results
%*********************************************************************