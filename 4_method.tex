%*********************************************************************
%	PROPOSED METHODOLOGY
%*********************************************************************
\chapter{Proposed work plan and activities}
\label{ch:method}
%---------------------------------------------------------------------
%Detailed breakdown of proposed work packages.

% Your methodology or research approach should be clearly described. 
% It must also discuss how you will validate your results. It must 
% be clear to the doctoral degrees board how and why you are following 
% the methodology you have presented.
%---------------------------------------------------------------------
% Taken from "Scientific Writing by D. Branch Moody"
%---------------------------------------------------------------------
% A good methods section would allow an expert in the same field to 
% reproduce your experiments using only your methods section and the 
% papers you cite in this section. 
%
% The approach to methods section is changing rapidly because online
% publishing is causing most journals to have two methods sections,
% with the Main Methods appearing in short form in the printed 
% paper and a more lengthy Supplemental Methods described in an 
% online section.  This division is occurring because some readers 
% of the paper need to know every detail so they can reproduce 
% the result, whereas non-specialists merely want to know some 
% general information about how the experiments were done. For 
% example, the strict word limits at PNAS create a situation in 
% which nearly all of the detailed methods are published in the
% supplemental online section.  It is very important to read 
% the Instructions to Authors before starting to write, so as 
% to understand whether a long, short or split methods section
% is needed.
%
% To meet the strict length limits yet still meet the "sufficient to
% reproduce" standard, a good practice is to really detail every 
% aspect of any new method reported for this first time in this 
% paper, and provide citations for previously published methods 
% along with an "as reported previously" statement.  
%
% Carefully listing the origin and precise name of every chemical, 
% cell, monoclonal, mouse, gene construct or other specialized reagent
% is particularly important, as no experiment can be reproduced 
% without this information. 
%---------------------------------------------------------------------
\section{Work packages}
\label{sec:workpackages}

\begin{table}[H]
    % \label{table:workpackage1}
    \centering
    \renewcommand{\arraystretch}{2} % Adjust this value to set the minimum row height
    \arrayrulecolor{tablerule} % Set the color of the table lines
    \begin{tabular}{|p{0.15\textwidth}|p{0.6\textwidth}|}
        \hline
        \rowcolor{titlebg}\multicolumn{2}{|l|}{\textbf{\stepcounter{workpackage}\theworkpackage: Work package title}}          \\
        \hline
        \textbf{\cellcolor{colbg}\textcolor{defaulttext}{Aim}}         &
        %----------------------------------------------------------------   
        Placeholder text here                                                                                                  \\
        %---------------------------------------------------------------- 
        \hline
        \textbf{\cellcolor{colbg}\textcolor{defaulttext}{Timeline}}    &
        %----------------------------------------------------------------   
        Placeholder text here                                                                                                  \\
        %---------------------------------------------------------------- 
        \hline
        \rowcolor{objectivebg}\multicolumn{2}{|l|}{\textbf{\stepcounter{objective}\theobjective: Objective title}}             \\
        \hline
        \textbf{\cellcolor{colbg}\textcolor{defaulttext}{Description}} &
        %----------------------------------------------------------------   
        Placeholder text here                                                                                                  \\
        %----------------------------------------------------------------     
        \hline
        \multicolumn{2}{|p{\dimexpr\textwidth-2\tabcolsep}|}{\footnotesize\textbf{Activities} \begin{task}
                                                                                                  \item Placeholder text for Task 1
                                                                                                  \item Placeholder text for Task 2
                                                                                                  \item Placeholder text for Task 3
                                                                                              \end{task}} \\
        \hline
        \rowcolor{objectivebg}\multicolumn{2}{|l|}{\textbf{\stepcounter{objective}\theobjective: Objective title}}             \\
        \hline
        \textbf{\cellcolor{colbg}\textcolor{defaulttext}{Description}} &
        %----------------------------------------------------------------   
        Placeholder text here                                                                                                  \\
        %----------------------------------------------------------------     
        \hline
        \multicolumn{2}{|p{\dimexpr\textwidth-2\tabcolsep}|}{\footnotesize\textbf{Activities} \begin{task}
                                                                                                  \item Placeholder text for Task 1
                                                                                                  \item Placeholder text for Task 2
                                                                                                  \item Placeholder text for Task 3
                                                                                              \end{task}} \\
        \hline
    \end{tabular}
\end{table}

%---------------------------------------------------------------------
\begin{table}[H]
    \label{table:workpackage1}
    \centering
    \renewcommand{\arraystretch}{2} % Adjust this value to set the minimum row height
    \arrayrulecolor{tablerule} % Set the color of the table lines
    \begin{tabular}{|p{0.15\textwidth}|p{0.7\textwidth}|}
        \hline
        \rowcolor{titlebg}\multicolumn{2}{|l|}{\textbf{\stepcounter{workpackage}\theworkpackage: Work package title}}          \\
        \hline
        \textbf{\cellcolor{colbg}\textcolor{defaulttext}{Aim}}         &
        %----------------------------------------------------------------   
        Placeholder text here                                                                                                  \\
        %---------------------------------------------------------------- 
        \hline
        \textbf{\cellcolor{colbg}\textcolor{defaulttext}{Timeline}}    &
        %----------------------------------------------------------------   
        Placeholder text here                                                                                                  \\
        %---------------------------------------------------------------- 
        \hline
        \rowcolor{objectivebg}\multicolumn{2}{|l|}{\textbf{\stepcounter{objective}\theobjective: Objective title}}             \\
        \hline
        \textbf{\cellcolor{colbg}\textcolor{defaulttext}{Description}} &
        %----------------------------------------------------------------   
        Placeholder text here                                                                                                  \\
        %----------------------------------------------------------------     
        \hline
        \multicolumn{2}{|p{\dimexpr\textwidth-2\tabcolsep}|}{\footnotesize\textbf{Activities} \begin{task}
                                                                                                  \item Placeholder text for Task 1
                                                                                                  \item Placeholder text for Task 2
                                                                                                  \item Placeholder text for Task 3
                                                                                              \end{task}} \\
        \hline
        \rowcolor{objectivebg}\multicolumn{2}{|l|}{\textbf{\stepcounter{objective}\theobjective: Objective title}}             \\
        \hline
        \textbf{\cellcolor{colbg}\textcolor{defaulttext}{Description}} &
        %----------------------------------------------------------------   
        Placeholder text here                                                                                                  \\
        %----------------------------------------------------------------     
        \hline
        \multicolumn{2}{|p{\dimexpr\textwidth-2\tabcolsep}|}{\footnotesize\textbf{Activities} \begin{task}
                                                                                                  \item Placeholder text for Task 1
                                                                                                  \item Placeholder text for Task 2
                                                                                                  \item Placeholder text for Task 3
                                                                                              \end{task}} \\
        \hline
    \end{tabular}
\end{table}

%---------------------------------------------------------------------
\section{Timeline}
\label{sec:Timeline}
%---------------------------------------------------------------------




%*********************************************************************
% End of Methodology
%*********************************************************************