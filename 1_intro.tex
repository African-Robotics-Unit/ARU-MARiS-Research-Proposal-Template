%*********************************************************************
%	INTRODUCTION TO RESEARCH PROPOSAL
%*********************************************************************
\chapter{Introduction}
\label{ch:intro}
%---------------------------------------------------------------------
% Give a brief background to the focus of your work. Highlight 
% why the work is significant and how it will contribute to your 
% field. It must be clear why the work is being conducted.
%
% Each research question should generate a mini-hypothesis and 
% lead to one research aim. Each aim will then form the main 
% chapters of your thesis. For a PhD thesis each of these aims 
% should be planned with a paper output in mind. The research 
% questions should be tightly linked to your overall hypothesis. 
% The hypothesis should be supported up by your literature research. 
%
% A PhD thesis must be novel work, however an MSc project only needs 
% to demonstrate advanced mastering of a particular technique or 
% research area.
%
% Outline the scope of the work. Keep your timeline and resources 
% in mind here. Be realistic. A PhD thesis should not take longer 
% than 2.5 years, ideally, if you plan to submit in August of your 
% 3rd year (6 mon research proposal and preliminary investigation; 
% 1.5 yr experimental work; 6 mon write up) and an MSc project 
% should not take longer than 18 months, ideally, if you wish to 
% submit in August of your 2nd year (3 mon preliminary investigation 
% and proposal; 1 yr experimental work; 3 mon write up).
%
% Plan your paper outputs at this stage. Look for peer-reviewed 
% journals-conferences that are relevant to your field of research 
% and consider how to write up your work as a paper. If you expect 
% to produce an artefact-product also list these outputs. Be realistic, 
% it is better to have fewer papers of high quality and impact than 
% more poor quality papers.
%---------------------------------------------------------------------
% Taken from "Scientific Writing by D. Branch Moody"
%---------------------------------------------------------------------
% The Introduction appears in the manuscript prior to the results 
% and methods section, yet this guide explains that it is better to 
% write the results and methods first.  This is no accident. The 
% results and methods are the substantial core of the paper so it 
% is helpful to have a really concrete description of the data before 
% the writing the more subjective sections that connect to the larger
% literature. 
%
% The introduction usually starts with a significance statement - an 
% idea that nearly everyone would accept as clearly important. 
% Example: "T cells play a key role in immune response."  Next 
% comes a statement of missing knowledge followed by a general 
% description of the results of the manuscript.  As with paper 
% titles, the main goal is to draw the reader in.   Writing 
% introductions requires creativity, such that the approach to 
% every paper is somewhat different.  Yet a few reliable and broadly
% applicable suggestions follow:
%
% First sentence.  It should have punch and say something important 
% and/or surprising.  
%
% Keep it short.  Five or fewer paragraphs is best and at some short 
% form journals you only get one half of one paragraph. 
%
% Head off confusion later.  Introduce any key reagents or 
% experimental systems that are unfamiliar to most readers. 
%
% Last paragraph of the introduction.  By tradition, this is 
% usually an overview of the findings starting with "Here we 
% report?"  The most common mistake is to preview each of the 
% three or four key findings of the paper.  This is unnecessary 
% because the reader has just read the abstract one page earlier, 
% so a detailed listing of results becomes redundant (and also gives 
% away the surprise of the story.)  Instead the authors can point 
% to a general problem that is solved here using broader language.
% Example, "Whereas most studies in the field have emphasized 
% the separate roles of Toll like receptors and inflammasome 
% activation, our results carried out with mouse macrophages in vitro 
% and in vivo, identify stimuli that effective activate both TLRs 
% and the inflammasome."  In this case, the identity of the 
% stimulus and stepwise approach to discovering it are not needed 
% in the introduction section because the results section will 
% explain this in detail. 
%=====================================================================
\section{Problem statement}
\label{sec:problemstatement}
%The research problem statement refers to an area of concern, a condition to be improved upon, a challenge to be explored, a difficulty to be eliminated, or a scholarly question the requires further interrogation to ground theory and praxis, that points to the need for meaningful understanding and deliberate investigation.

%=====================================================================
\section{Research questions and hypothesis}
\label{sec:hyphothesis}
%---------------------------------------------------------------------
%link these to your aims and work packages


From these questions I propose that:
\begin{guess*}
	\begin{minipage}[t]{5 in}
		Hypothesis
	\end{minipage}
\end{guess*}


%=====================================================================
\section{Scope, assumptions and limitations}
\label{sec:scope}


%---------------------------------------------------------------------
%=====================================================================
\section{Potential impact and outputs}
\label{sec:outputs}
%---------------------------------------------------------------------
%•	Impact refers to both the tangible and intangible influence derived and/or caused by the research outcomes/outputs. 
%•	Impact statements indicate what the researcher hopes to achieve, without introducing any bias, through their research (i.e., impact on Global Change and/or Bio-economy, etc.).

%*********************************************************************
% End of Introduction
%*********************************************************************
